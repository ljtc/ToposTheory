\chapter*{Notas al Lector}
\label{chap:notas_al_lector}
\addcontentsline{toc}{chapter}{Notas al Lector}

A lo largo de este libro, us sólo sistema de numeración es usado para
definiciones, lemas, teoremas, observaciones, etc.; 
el número \(n.p.q\) normalmente denota la \(q\)-ésima referencia numerada en la
sección \(p\) del capítulo \(n\). \pend{Por el momento he cambiado el estilo de
numeración por las razones que se comentan en el libro, parece no tener una buena
lógica. Por ejemplo, 8.20 es la decima referencia en la sección 8.1. Johnstone
afirma que no hy un número que refiera a dos referencias, aún así preferí usar
el sistema de numeración que usa {\LaTeX} por defecto.}

Al final de cada capítulo se encontrarán algunos ejercicios: cerca de diez en
cada uno de los primeros capítulos, más en los últimos. Ellos varías
considerablemente en dificultad, algunos son completamente rutinarios, mientras
otros son un tanto sustanciales. No he dado ninguna indicación de cuáles
ejercicios considero más fáciles (el orden es el del material del capítulo al
cual refieren), pero he dado muchas pistas en la mayoría de los difíciles. En
varios casos, el resultado de un ejercicio es usado ya sea en los ejercicios o
el texto de un capítulo posterior; estos ejercicios se distinguen con una daga
(\dagger).

El siguiente resumen de interdependencia lógica de varios capítulos puede ser
útil para el lector que esté interesado en un tópico en particular. El
capítulo~\ref{cap:0} contiene un resumen de cierto material de fondo que es
requerido para motivar la definición de topos, o para dar una fuente de
ejemplos. Los capítulos~\ref{cap:1}--\ref{cap:5} forman el núcleo del libro; de
estos capítulos \ref{cap:1}--\ref{cap:4} siguen un camino más o menos geodésico
(con algunas desviaciones como la sección~\ref{sec:4.2}) de la definición de
topos (\ref{1.11}) al teorema relativo de Giraud (\ref{4.46}) y la existencia de
productos fibrados en \(\betop\) (\ref{4.48}). La dependencia lógica entre estos
cuatro capítulos es entonces bastante cercana a ser un orden lineal.

Sin embargo, la mayoría del material del capítulo~\ref{cap:2} (sobre categoría
internas) es algo técnico, y algunos lectores lo encontrar difícil en la primera
lectura. Recomiendo a esos lectores que omitan todo el capítulo~\ref{cap:2},
excepto por el teorema~\ref{2.32} (el cual es importante, y tiene aplicaciones
en otras áreas además de teoría de categorías internas), y continúen con el
capítulo~\ref{cap:3}. (Hay algunas referencias al capítulo~\ref{cap:2} en la
sección~\ref{sec:3.3}, pero puedes regresar a ellas cuando sea necesario.) Luego
puedes continuar con la primera sección del capítulo~\ref{cap:4}, todo el
capítulo~\ref{cap:5}, excepto por algunas partes de la sección~\ref{sec:5.3}, e
incluso las primeras dos secciones del capítulo~\ref{cap:6} antes de regresar al
capítulo~\ref{cap:2}.

El capítulo~\ref{cap:5} introduce un gran número de conceptos los cuales, aunque
son parte de la corriente principal de la teoría de topos, no se involucran en
la demostración del teorema de Giraud relativo. En particular, contiene una
descripción del lenguaje interno de un topos, el cual es usado libremente en la
segunda mitad del libro.

Los últimos cuatro capítulos presentan varias extensiones y aplicaciones de la
teoría básica; originalmente esperaba hacerlos lógicamente independientes, de
tal forma que pudieran ser leídos en cualquier orden, pero inevitablemente
algunos cruces y conexiones se han establecido entre ellos.. La siguiente tabla
resume los importantes:
\begin{center}
\begin{tabular}{ll}
  \toprule
  Antes de leer & se recomienda leer \\
  \midrule
  \ref{sec:7.4} & \ref{sec:6.3} y \ref{sec:6.5} \\
  \ref{sec:8.1} & \ref{sec:7.5} \\
  \ref{sec:8.4} & \ref{sec:6.2} \\
  \ref{sec:9.1} & \ref{sec:6.2} y \ref{sec:6.4} \\
  \bottomrule
\end{tabular}
\end{center}
Hay mś conexiones entre los ejercicios de estos últimos cuatro capítulos (ver,
por ejemplo, los ejercicios~\ref{ej:6.11}, \ref{ej:8.7}, \ref{ej:9.6} y
\ref{ej:9.14}).

El apéndice es una presentación de un material que originalmente se pensaba
incluir en el capítulo~\ref{cap:2}; se removió de ahí porque, parece seguro que
pronto se convertirá en parte de la corriente principal de la teoría de los
topos, es posible que la definición básica de \enquote{categoría localmente
interna} aún no ha alcanzado su forma final. Puede ser leído en cualquier
momento después del capítulo~\ref{cap:2}; aunque hace muchas referencias a
capítulos siguientes.

A lo largo del libro, las referencias a la bibliografía están encerradas entre
corchetes. La bibliografía está dividida en cuatro secciones: la sección A
consiste de \enquote{referencias estándar} a otras áreas de las matemáticas (por
ejemplo, a teoría de retículas o topología algebraica) que son usadascuando un
teorema o definición de esa área es citado en el texto. La sección B tiene
trabajos de naturaleza general sobre teoría de topos, y algunos artículos
introductorios escritos para lectores no especialistas (por ejemplo,
\pend{[BM]} y \pend{[WI]}). La sección C tiene el resto de las referencias sobre
teoría de topos y un número de artículos cercanamente relacionados sobre teoría
de categorías, teoría de gavillas, etc. Las secciones B y C juntas tienen como
objetivo presentar una lista completa de los artículos publicados hasta ahora en
teoría de topos; sin embargo, no incluyo resúmenes de pláticas, ni tesis de
doctorado a menos que tengan resultados importantes no publicado en ningún otro
lado. La sección D tiene el resto de las referencias citadas en el texto. Las
referencias en las secciones A y B están indicadas por un código de dos letras;
aquellas en las secciones C y D están enumeradas consecutivamente. En las cuatro
secciones, he indicado los \textit{Mathematical Reviews} y números, donde ellos
están.