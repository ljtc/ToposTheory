\chapter*{Prefacio}
\label{chap:prefacio}
\addcontentsline{toc}{chapter}{Prefacio}

Los orígenes de este libro se pueden rastrear hasta una serie de seis
seminarios, los cuales di en Cambridge en el invierno de 1973/74, y que forman
el núcleo de los capítulos~\ref{cap:1}--\ref{cap:6} del presente. Más seminarios
en la misma serie, cubriendo partes de los capítulos~\ref{cap:0}, \ref{cap:7} y
\ref{cap:9}, fueron dados por Barry Tennison y Robert Seely. Por petición
popular, las notas de esos seminarios fueron escritas y tuvieron una circulación
limitada. En el verano de 1974, empecé a revisar y expandir esas notas, con la
idea de que podrían formar un libro en algún día. Durante el invierno y
primavera de 1975, mientras estaba en la Universidad de Liverpool, pude dar un
curso cubriendo el material de los capítulos~\ref{cap:0}--\ref{cap:5} y
\ref{cap:8} con algo de detalle. Al final de este período, tenía una imagen
suficientemente clara de la forma en general de este libro; y (motivado por
Michael Butler) empecé la escritura de este en julio de 1975. De octubre de 1975
a marzo de 1976 estaba en la Universidad de Chicago, donde había un seminario
semanal sobre teoría de topos organizado por Saunders Mac~Lane y yo; el material
cubierto durante este período se extrajo principalmente de los capítulos
\ref{cap:2}, \ref{cap:4}, \ref{cap:5}, \ref{cap:6} y \ref{cap:9}, y los
expositores (además de mí) fueron Kathy Edwards, Steve Harris y Steve Landsburg.
También durante este período, escribí el texto de los
capítulos~\ref{cap:2}--\ref{cap:5} y la mayoría del capítulo~\ref{cap:6}; el
resto del texto fue completado durante mayo--junio de 1976 después de mi regreso
a Cambridge.

Las lecturas y seminarios mencionadas arriba tuvieron una influencia muy directa
en el texto del libro, y todos aquellos que asistieron a ellos (en particular
aquellos cuyos nombres aparecen arriba) merecen mi agradecimiento por la papel
que jugaron en su formación. También me he beneficiado de contactos informales
con muchos matemáticos en conferencias y otros lados. Entre aquellos cuyas
(mayormente no publicadas) ideas he tomado con mucho gusto prestadas están
Julian Cole, Radu Diaconescu, Mike Fourman, Peter Freyd, André Joyal y Chris
Mulvey. John Gray me dio valiosos concejos sobre las partes \(2\)-categóricas, y
Jack Duskin y Barry Tennison me ayudaron a mejorar mi entendimiento sobre
cohomología. Debo agradecer a Jean Bénabou por las muchas ideas que he tomado
prestadas consciente o inconscientemente de él, y  a Tim Brook por la
compilación de la bibliografía.

Los cuatro matemáticos restantes con quienes estoy en deuda y deben ser
mencionados individualmente son los siguientes. Myles Tierney me introdujo a la
teoría de topos a través de sus lecturas en Varenna en 1971; buscando la versión
sin publicar \pend{[TV]}, encuentro increíble que me haya enseñado tanto en ocho
lecturas cortas. La ayuda de Gavin Wraith y su ánimo han significado mucho para
mí y sus lecturas en Bangor \pend{[WB]} sirvieron como modelo para algunas
partes de este libro. Como cualquier otra persona que trabaje en teoría de
topos, tengo una deuda abrumadora con Bill Lawvere, por sus puntos de vista
precursores; también me he beneficiado en un nivel más personal de sus ideas y
conversaciones. Sobre todo, debo expresar mi deuda con Saunders Mac~Lane: pero
para él nunca debí convertirme en un teórico de topos en primer lugar; y el
cuidado con el que leyó el manuscrito original y dio sugerencias para mejorar en
casi cada párrafo, en conjunto ha sido fuera de lo común. Si aún permanece algún
error o algo que no sea claro en el texto, son seguramente testimonio de mi
perversidad en lugar de su falta de cuidado.

En un nivel diferente, pero no menos importante, debo agradecer a las
Universidades de Liverpool y Chicago, además de St John's College y Cambridge
por emplearme durante la escritura del libro; Paul Cohn, por aceptarlo aceptarlo
para su publicación en la serie L.M.S Monographs; y al equipo del Academic Press
por la eficiencia con la cual transformaron mi manuscrito amateur en el libro
que ves frente a ti.

\vspace{3ex}

\noindent
Cambridge, Junio de 1977