\chapter*{Introducción}
\label{chap:intro}
\addcontentsline{toc}{chapter}{Introducción}

La teoría de topos tiene sus orígenes en dos líneas separadas del desarrollo
matemático, las cuales permanecieron separadas durante diez años. Para tener una
apreciación balanceada de la importancia del tema, creo necesario
considerar la historia de estas dos líneas, y entender porque se unieron cuando
lo hicieron. Por lo tanto, inicio esta introducción con un pasaje histórico
(personal, y sin duda muy sesgado).

La primera de las dos líneas comienza con el surgimiento de la \emph{teoría de
gavillas}, originada en 1945 por J. Leray, desarrollada por H. Cartan y A. Weil
entre otros, y culminando en el trabajo publicado por J. P. Serre \pend{[107]}, 
A. Grothendieck \pend{[42]} y R. Godement \pend{[TF]}. Como una buena parte del
álgebra homológica, la teoría de gavillas originalmente fue concebida como una
herramienta para la topología algebraica, para axiomatizar la noción de
\enquote{sistema local de coeficientes} el cual era esencial para una buena
teoría de cohomología de espacios que no son simplemente conexos; y el título
completo del libro de Godement índica que aún era vista así en 1958. Sin
embargo, antes de esta fecha, la potencia de la teoría de gavillas había sido
reconocida por geómetras algebraicos y analíticos; y en años más recientes, su
influencia se ha extendido a muchas otras áreas de las matemáticas. (Para tener
dos ejemplos bastante diferentes ver \pend{[49]} y \pend{[106]}.)

Sin embargo, en geometría algebraica fue descubierto rápidamente que la noción
topológica de gavilla no era del todo adecuada, ya que la única topología
disponible en variedades algebraicas abstractas o esquemas, la topología Zariski,
no tenía \enquote{suficientes abiertos} para proveer de una buena noción
geométrica de localización. En su trabajo sobre técnicas de descenso
\pend{[43]} y el grupo fundamental étale \pend{[44]}, A. Grothendieck observó
que reemplazar \enquote{inclusión abierta de Zariski} por \enquote{morfismo
étale} era un paso en la dirección correcta; pero desafortunadamente los
esquemas que son étale sobre un esquema dado en general no forman un conjunto
parcialmente ordenado. Fue entonces necesario inventar la noción de
\enquote{topología de Grothendieck} sobre una categoría arbitraria, y la noción
generalizada de gavilla para tal topología para dar un marco para el desarrollo
de la cohomología étale.

Este marco fue construido durante el \enquote{Seminaire de Géométrie Algébrique
du Bois Marie} durante 1963--64 por Grothendieck con la asistencia de M. Artin,
J. Giraud, J. L. Verdier, y otros. (Las actas de este seminario fueron
publicadas en una versión muy alargada \pend{[GV]}, incluyendo algunos
resultados notables de P. Deligne, ocho años después.) Entre los resultados más
importantes del seminario original fue el teorema de Giraud, que muestra que las
categorías de gavillas generalizadas que surgen de esta manera pueden ser
completamente caracterizadas por propiedades de exactitud y condiciones de
tamaño; a la vista de este resultado, rápidamente se hizo evidente que estas
categorías de gavillas eran un tema de estudio más importante que los sitios (=
categorías + topologías) que les dan origen. A la vista de esto y dado que una
categoría con una topología puede ser vista como un \enquote{espacio topológico
generalizado}, el (ligeramente desafortunado) nombre de \emph{topos} fue dado a
las categorías que satisfacen los axiomas de Giraud.

No obstante, los topos seguían considerándose como vehículos primarios para
acarrear teorías de cohomología; no sólo cohomología étale, sino también la
\enquote{fppf}, cohomologías cristalinas, entre otras. La potencia de la
maquinaria desarrollada por Grothendieck fue ampliamente demostrada por los
sustanciales resultados geométricos obtenidos usando estas teorías de
cohomología en los años siguientes, culminando en la demostración de P. Deligne
\pend{[149]} de las famosas \enquote{conjeturas de Weil} ---el análogo \(\mod p\) de
la hipótesis de Riemann---. La maquinaria en sí fue desarrollada aún más, por
ejemplo en el trabajo de J. Giraud \pend{[38]} en cohomología no abeliana. Pero
el significado completo de la sentencia \enquote{el topos es más importante que
el sitio} parece que nunca fue apreciado por la escuela de Grothendieck. Por
ejemplo, aunque eran conscientes de la estructura cartesiana cerrada de los
topos (\pend{[GV, IV 10]}), nunca explotaron al máximo esta idea siguiendo las líneas
marcadas por Eilenberg y Kelly \pend{[160]}. Fue, por lo tanto, necesario que una
segunda línea de desarrollo proveyera el ímpetu para la teoría elemental de
topos.

El punto de partida de esta segunda línea se considera generalmente que es el
artículo pionero de F. W. Lawvere de 1964 sobre la teoría elemental de la
categoría de conjuntos \pend{[71]}. Sin embargo, considero que es necesario ir
un poco más atrás, a la demostración del teorema del encaje de Lubkin, Heron
Freyd y Mitchell para categorías abelianas \pend{[AC]}. Fue este teorema el
cual, mostrando que hay un conjunto explícito de axiomas elementales que
implican todas las propiedades de exactitud (finitas) de categorías de módulos,
pavimento el camino para un verdadero desarrollo autónomo de la teoría de
categorías como fundamento de las matemáticas.

(Casualmente el teorema del encaje de Freyd y Mitchell se considera
frecuentemente como una culminación en lugar de un punto de partida; esto es
porque me parece una mala interpretación (o al menos una inversión) de su
verdadero significado. Comunmente se piensa que dice \enquote{si quieres
demostrar algo acerca de una categoría abeliana, puedes asumir que es una
categoría de módulos}; mientras que yo creo que su verdadera importancia es
\enquote{si quieres demostrar algo acerca de categoría de módulos, puedes
trabajar en una categoría abeliana en general} ---el teorema del encaje asegura
que tu resultado será válido en esta generalidad, y olvidando la estructura
explícita de la categoría de módulos serás forzado a concentrarte en los
aspectos esenciales del problema---. Como ejemplo, compara la demostración
módulo teórica del lema de la serpiente en \pend{HA} con la demostración en
categorías abelianas en \pend{[CW]}.)

Este teorema pronto fue seguido por el artículo de Lawvere \pend{[71]},
estableciendo una lista de axiomas elementales los cuales, agregando los axiomas
no elementales de completud y pequeñez local, son suficientes para caracterizar
a la categoría de conjuntos. (En un artículo subsecuente \pend{[72]}, Lawvere
provee una axiomatización similar para la categoría de categorías pequeñas, y D.
Schlomiuk \pend{[105]} hizo lo mismo para la categoría de espacios topológicos.)

Uno bien puede preguntar por qué este artículo no fue seguido inmediatamente por
la explosión de actividad que supuso la introducción de los topos elementales
seis años después. En retrospectiva, la respuesta es que los axiomas de Lawvere
son demasiado especializados: la categoría de conjuntos es un objeto
extremadamente útil como fundamento de las matemáticas, pero como objeto de
estudio axiomático no es (¡\textit{ritmo} de la actividad de Martin, Solovay, et.
al!) tremendamente interesante ---es demasiado \enquote{rígida} como para tener
una estructura interna---. De manera similar, si los axiomas de categorías
abelianas hubieran aplicado sólo a la categoría de grupos abelianos, y no a la
categoría de módulos o a la de gavillas abelianas, ellos también habrían sido
rechazados. Así, lo que se necesitaba para la categoría de conjuntos era una
axiomatización que también cubriera a las categorías de functores con valores en
conjuntos y categorías de gavillas con valores en conjuntos ---es decir, los
axiomas de topos elementales---.

En sus artículos subsecuentes (\pend{[73] y [75]}), Lawvere comienza a
investigar la idea de que el conjunto con dos elementos \(\set{\true, \false}\)
puede ser visto como un \enquote{objeto de valores de verdad} en la categoría de
conjuntos; en particular, observó que la presencia de tal objeto en una
categoría arbitraria nos permite reducir al axioma de comprensión a un enunciado
elemental acerca de funtores adjuntos. La misma idea es el corazón del trabajo
de H. Volger (\pend{[125] y [126]}) sobre lógica y categorías semánticas.

Mientras tanto, el lado del teorema del encaje de esto fue desarrollado por M.
Barr \pend{[2]}, quien formuló la  noción de \emph{categoría exacta} y la usó
como base para un teorema de encaje no aditivo. La noción cercanamente
relacionada de \emph{categoría regular} fue formulada independientemente por P.
A. Grillet \pend{[41]} y D. H. Van Osdol \pend{[122]}, quienes la usaron en sus
investigaciones sobre teoría general de gavillas; el mismo Barr obsrevó que el
teorema de Giraud puede ser visto como algo más que un caso especial de este
teorema de encaje. Esto quizás represente (lógicamente, si no cronológicamente)
el primer acercamiento de las dos líneas de desarrollo mencionadas antes.

Sin embargo, cerca del mismo tiempo la atención de Lawvere dió un giro hacia los
topos de Grothendieck; él observó que todo topos de Grothendieck tiene un objeto
de valores de verdad \(\Omega\), y que la noción de topología de Grothendieck
está cercanamente conectada con endomorfismos de \(\Omega\) (ver \pend{[LH]}).
Durante los años 1969--70, Lawvere y M. Tierney (quien antes había contribuido a
la teoría de categorías exactas) comenzaron a investigar las consecuencias de
tomar \enquote{existe un objeto de valores de verdad} como un axioma; el
resultado fue la teoría elemental de topos. Una proporsión notablemente grande
de la teoría básica fue desarrollada en ese período de 12 meses, como se hará
evidente del gran número de teoremas en los capítulos 1--4 de este libro en cuya
demostración se hará referencia a Lawvere y Tierney. 

Una vez que estos teoremas fueron conocidos por los matemáticos en general (es
decir, después de las lecturas de Lawvere en Zürich y Niece \pend{[LN]} en el
verano de 1970 y la conferencia en Dalhousie \pend{[LH]} en enero de 1971),
fueron inmediatamente aceptados y desarrollados por varias personas. Uno de los
primeros y más importantes fue P. Freyd, cuyas lecturas en University of New
South Wales \pend{[FK]} exploraron la teoría de encajes de topos; en
retrospectiva esto parece haber sido algo como un callejón sin salida, de que la
inversión del metateorema usual, mencionado arriba respecto a categorías
abelianas, aplica con incluso más fuerza en la teoría de topos ---ya que la gran
virtud de los axiomas de topos es su carácter elemental, uno no tiene que apelar
a un teorema de encaje no elemental para demostrar hechos elementales acerca de
topos. (El teorema de encaje de Freyd no será encontrado en este libro; pero la
parte más importante (y elemental) de él que muestra que todo topos puede ser
encajado en un topos booleano, es demostrado en \autoref{sec:7.5}). No obstante,
el trabajo de Freyd contiene muchos resultados técnicos importantes; en
particular su teorema de caracterización de objetos de números naturales es de
suma importancia.