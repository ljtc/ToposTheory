\chapter{Preliminares}
\label{cap:0}

\section{Teoría de categorías}
\label{sec:0.1}

Esta sección no tiene la intención de dar una introducción detallada a las ideas
básicas de la teoría de categorías; nuestra intensión más bien es indicar
algunos de los conceptos y teoremas que serán asumidos como familiares, y
establecer algunas nociones estándar. El lector que no se considere familiarizado
con estos conceptos básicos será bien aconsejado a consultar el excelente libro
de Mac~Lane \pend{[CW]}, o cualquier otro texto estándar sobre teoría de
categorías antes de proceder más adelante con este libro.

Normalmente usaremos letras mayúsculas en script 
(\(\lcatC\), \(\lcatD\), \(\lcatE\),\dots) para denotar categorías
\enquote{grandes}. Cuando afirmemos \enquote{\(\lcatC\) es una categoría} sin
ningún calificativo adicional, se entenderá que \(\lcatC\) es un modelo de la
\enquote{teoría elemental de categorías} \pend{[72]}, es decir, \(\lcatC\) es
una \emph{metacategoría} en el sentido de \pend{[CW], capítulo I}. Esto
significa que no pensamos a \(\lcatC\) siendo formalmente definida dentro de un
modelo particular de la teoría de conjuntos; en particular, si \(X\) y \(Y\) son
objetos de \(\lcatC\) no pedimos que los morfismos de \(X\) a \(Y\) en
\(\lcatC\) formen un conjunto.

Sin embargo, a excepción del capítulo~\ref{cap:9}, normalmente asumiremos que
tenemos un modelo (fijo) de la teoría de conjuntos (incluyendo el axioma de
elección cuando sea necesario); y usaremos la letra \(\con\) para denotar la
\emph{categoría de conjuntos y funciones} que obtenemos de él. Usamos el término
\emph{categoría pequeña} para una categoría cuyos morfismos forman un conjunto.
Si \(\catC\) es una categoría pequeña, escribimos \(\pge{\catC}\) para la
categoría de \emph{pregavillas} sobre \(\catC\), es decir, funtores
contravariantes de \(\catC\) en \(\con\); entre los objetos de \(\pge{\catC}\),
tenemos los funtores \emph{representables} \(\gy_U\), donde \(U\) es un objeto
de \(\catC\), definido como \(\gy_U(V)=\hom_{\catC}(V,U)\). (Por razones
tipográficas, a veces escribiremos \(\gy(U)\) en lugar de \(\gy_U\); además
escribiremos \(\gy^U\) para el funtor covariante representable
\(\hom_{\catC}(U,-)\).) Frecuentemente haremos uso de los siguientes dos
resultados:

\begin{lemma}[label={0.11},note={Yoneda \pend{[187]}},short-note={Yoneda}]
  Dados \(U\) y \(X\) objetos de \(\catC\) y \(\pge{\catC}\) respectivamente,
  hay una biyección (natural en las dos variables) entre morfismos \(\gy_U\to X\)
  en \(\pge{\catC}\) y elementos del conjunto \(X(U)\).
\end{lemma}

\begin{lemma}
  Cualquier objeto de \(\pge{\catC}\) se puede expresar como un colímite de un
  diagrama cuyos vértices son funtores representables.
\end{lemma}
\begin{proof}
  Sea \(X\) un objeto de \(\pge{\catC}\), y sea \((\comma{\catC,X})\) la
  \emph{categoría coma} (pequeña) cuyos objetos son pares \(U,\alpha\), con
  \(U\) un objeto de \(\catC\) y \(\alpha\colon\gy_U \to X\) en \(\pge{\catC}\),
  y cuyos morfismos son triángulos conmutativos
  \begin{equation*}
    \begin{tikzcd}
      \gy_U \ar{rr}\ar{rd} && \gy_V\ar{ld} \\
      & X
    \end{tikzcd}
  \end{equation*}
  en \(\pge{\catC}\).

  Así, tenemos un diagrama obvio \((\comma{\catC,X})\to\pge{\catC}\) obvio que
  \enquote{olvida} dado por \((U,\alpha)\mapsto\gy_U\); se puede ver que el
  colímite de este diagrama es \(X\).
\end{proof}

Suponemos que el lector está familiarizado con las nociones de \emph{límite},
\emph{colímite} y \emph{funtores adjuntos}. (Usamos la notación \(T\dashv G\)
para \enquote{\(T\) es adjunto izquierdo de \(G\)}.) Aquí una advertencia:
cuando decimos que una categoría \emph{tiene límites} de un tipo particular
queremos decir que hay, para un diagrama del tipo apropiado, una elección
\emph{canónica} del límite. Por ejemplo, cuando decimos que \(\con\) tiene
productos binarios queremos decir que, dados dos conjuntos \(X\) y \(Y\), existe
bo sólo un conjunto cuyoe elementos están en biyección con los pares ordenados
\(\fami{x,y}\), sino un conjunto canónico, a saber el conjunto de todos los
pares ordenados. Sin embargo, cuando decimos que un funtor \emph{preserva
límites} no significa que preserva la elección canónica de los límites; y el
enunciado un objeto es \emph{un} límite de cierto diagrama no implica que es el
canónico.

Si una categoría \(\lcatC\) tiene objeto terminal (es decir, un límite para el
diagrama vacío) lo denotamos con \(1\); y si \(X\) es un objeto de \(\lcatC\),
también usamos \(X\) para denotar el único morfismo \(X\to 1\), y \(\id_X\) (o
simplemente \(\id\)) para el morfismo identidad en \(X\). (La confusión heredada
de esta notación puede ser justificada de manera imprecisa por el hecho de que
\(\id_X\) es el objeto terminal de la categoría \(\slice{\lcatC,X}\) de
\emph{objetos sobre} \(X\) y que si \(f\colon Y \to X\) es un objeto en esta
categoría, entonces \(f\) es también el único morfismo en \(\slice{\lcatC,X}\)
de \(f\) a \(\id_{1_X}\).) Si \(\lcatC\) tiene productos y/o productos fibrados,
usamos la letra \(\pi\) para denotar la proyección canónica de un producto o un
producto fibrado a uno de sus factores, con subíndice \(1\), \(2\),
\(3\),{\ldots} para denotar al primer, segundo, tercer,{\dots} factor. De manera
similar, usaremos comúnmente (pero no exclusivamente) la letra \(v\), con un
subíndice apropiado, para denotar la inclusión de un factor en un coproducto.

También asumimos familiaridad con la noción de \emph{mónada} (o triple),
\emph{comónada} y de álgebras sobre una \emph{mónada}. Haremos uso del
\enquote{teorema crudo de monacidad} de Beck \pend{[153]} es su forma de
\enquote{coigualador reflexivo}; recordar que un par paralelo 
\(f,g\colon X\to Y\) en una categoría \(\lcatC\) es \emph{reflexivo} si existe
\(h\colon Y\to X\) tal que \(hf = hg = \id_Y\). (En el caso \(\lcatC=\con\), esto
es equivalente a decir que la imagen de \((f,g)\colon X\to Y\times Y\) es una
relación reflexiva sobre \(Y\).)

\begin{theorem}[label=0.13]
  Sean \(F\colon\lcatC\to\lcatA\) y \(U\colon\lcatA\to\lcatC\) un par de
  funtores tales que \(F\dashv U\) y sea \(\monH\) la mónada sobre \(\lcatC\)
  inducida por esta adjunción. Supongamos que \(\lcatA\) tiene coigualadores de
  pares reflexivos, que \(U\) los preserva y que \(U\) refleja isomorfismos.
  Entonces \(U\) es monádico, es decir, el functor de comparación 
  \(K\colon\lcatA\to\talg{\lcatC}{\monH}\) es una equivalencia de categorías,
  donde \(\talg{\lcatC}{\monH}\) denota la categoría de \(\monH\)-álgebras.
\end{theorem}

También haremos uso de los siguientes teoremas acerca de categorías de álgebras:

\begin{theorem}[label=0.14,note={Eilenberg-Moore \pend{[162]}},
  short-note={Eilenberg-Moore}]
  Sea \(\monH=(H,\eta,\mu)\) una mónada sobre \(\lcatC\) y supongamos que \(H\)
  tiene adjunto derecho \(G\). Entonces existe una única estructura de comónada
  \(\monG=(G,\varepsilon,\delta)\) sobre \(G\) tal que la categoría 
  \(\coalg{\lcatC}{\monG}\) de \(\monG\)-coálgebras es isomorfa a
  \(talg{\lcatC}{\monH}\), con un isomorfismo que identifica a los dos funtores
  que olvidan.  
\end{theorem}

\begin{theorem}[label=0.15, 
  note={\enquote{teorema de levantamiento de adjuntos}, ver \pend{[54]}},
  short-note={levantamiento de adjuntos}]
  Sean \(\monH\) y \(\monK\) mónadas sobre categorías \(\lcatC\) y \(\lcatD\)
  respectivamente, \(T\colon\lcatC\to\lcatD\) un funtor y 
  \(\overline{T}\colon\talg{\lcatC}{\monH}\to\talg{\lcatD}{\monK}\) un funtor que es
  el \enquote{levantamiento} de \(T\) en el sentido que el cuadrado
  \begin{equation*}
    \begin{tikzcd}[ampersand replacement = \&]
      \talg{\lcatC}{\monH} \ar{r}{\overline{T}}\ar{d}[swap]{U} \&
      \talg{\lcatD}{\monK} \ar{d}{U} \\
      \lcatC \ar{r}[swap]{T} \& \lcatD
    \end{tikzcd}
  \end{equation*}
  conmuta, donde los \(U\) denotan a los funtores que olvidan. Supongamos
  también que \(\talg{\lcatC}{\monH}\) tiene coigualadores de pares reflexivos.
  Si \(T\) tiene adjunto izquierdo, entonces también lo tiene \(\overline{T}\).
\end{theorem}

\begin{theorem}[label=0.16, note={Linton \pend{[178]}}, short-note={Linton}]
  Sea \(\monH\) una mónada sobre una categoría \(\lcatC\) y supongamos que
  \(\talg{\lcatC}{\monH}\) tiene coigualadores de pares reflexivos. Si
  \(\lcatC\) tiene coproductos finitos (\(\con\)-indexados,
  respectivamente), entonces \(\talg{\lcatC}{\monH}\) tiene colímites finitos
  (\(\con\)-indexados, respectivamente).
\end{theorem}

Es claro de los teoremas \ref{0.13}, \ref{0.15} y \ref{0.16} que los
coigualadores de pares reflexivos (\enquote{coigualadores reflexivos}) juegan un
papel importante en la teoría de mónadas. Por lo tanto, es apropiado dar en este
punto un lema el cual, a pesar de ser capaz de una aplicación muy amplia (como
veremos en el capítulo~\ref{cap:6}), no ha encontrado su camino hacia los textos
estándar sobre teoría de categorías.

\begin{lemma}[label=0.17]
  Sea
  \begin{equation*}
    \begin{tikzcd}[ampersand replacement = \&, column sep=large]
      X_1 \ar[shift left]{r}{f_1}\ar[shift right]{r}[swap]{f_2}
          \ar[shift left]{d}{\alpha_2}\ar[shift right]{d}[swap]{\alpha_1} \&
      X_2 \ar{r}{f_3}
          \ar[shift left]{d}{\beta_2}\ar[shift right]{d}[swap]{\beta_1} \&
      X_3 \ar[shift left]{d}{\gamma_2}\ar[shift right]{d}[swap]{\gamma_1} \\
      Y_1 \ar[shift left]{r}{g_1}\ar[shift right]{r}[swap]{g_2}
          \ar{d}{\alpha_3} \&
      Y_2 \ar{r}{g_3} \ar{d}{\beta_3} \&
      Y_3 \ar{d}{\gamma_3} \\
      Z_1 \ar[shift left]{r}{h_1}\ar[shift right]{r}[swap]{h_2} \&
      Z_2 \ar{r}{h_3} \& Z_3
    \end{tikzcd}
  \end{equation*}
  un diagrama que satisface las condiciones \enquote{obvias} de conmutatividad en
  cualquier categoría (es decir, \(\beta_i f_j = g_j \alpha_i\) con \(i=1,2\),
  \(j=1,2\), etc.), en el que los renglones y columnas son coigualadores y los
  pares \((f_1,f_2)\) y \((\alpha_1,\alpha_2)\) son reflexivos. Entonces la
  diagonal
  \begin{equation*}
    \begin{tikzcd}[ampersand replacement = \&, column sep=large]
      X_1 \ar[shift left]{r}{\beta_1 f_1}\ar[shift right]{r}[swap]{\beta_2 f_2} \&
      Y_2 \ar{r}{\gamma_3 g_3} \& Z_3
    \end{tikzcd}
  \end{equation*}
  es un coigualador.
\end{lemma}
\begin{proof}
  Primero notamos que
  \begin{align*}
    \gamma_3 & = \coig(\gamma_1, \gamma_2)\\
             & = \coig(\gamma_1 f_3, \gamma_2 f_3) && \text{pues \(f_3\) es epi}\\
             & = \coig(g_3 \beta_1, g_3 \beta_2).
  \end{align*}
  Así, el cuadrado de abajo a la derecha es un coproducto fibrado; es decir, un
  morfismo \(\theta\colon Y_2\to T\) se factoriza a través de 
  \(\gamma_3 g_3\colon Y_2\to Z_3\) si y sólo si coiguala a \((g_1, g_2)\) y
  \((\beta_1,\beta_2)\). Si esa condición se satisface, entonces
  \(\theta\beta_1f_1=\theta\beta_2f_1=\theta g_1\alpha_2=\theta g_2\alpha_2
  =\theta\beta_2f_2\). Viceversa, si \(\theta\beta_1f_1=\theta\beta_2f_2\) y 
  \(s\colon X_2\to X_1\) es una escisión común para \(f_1\) y \(f_2\), entonces
  \(\theta\beta_1=\theta\beta_1f_1s=\theta\beta_2f_2s=\theta\beta_2\); y de
  manera similar \(\theta g_1=\theta g_2\). Así que \(Y_2\to Z_3\) es un
  coigualador de \(\beta_1 f_1\) y \(\beta_2 f_2\).
\end{proof}

\begin{comentario}
  No es necesario para la demostración que el cuadrado de abajo a la izquierda
  sea un coproducto fibrado. Si ya tenemos que un morfismo 
  \(\theta\colon Y_2\to T\) se factoriza a través de 
  \(\gamma_3 g_3\colon Y_2\to Z_3\) si y sólo si coiguala a \((g_1, g_2)\) y
  \((\beta_1,\beta_2)\),
  entonces se puede ver que \(\gamma_3g_3\) es un coigualador a partir de todas
  las conmutatividades y la escisión:
  \begin{equation*}
    \begin{tikzcd}[ampersand replacement = \&]
      X_1\ar[shift left]{r}{f_1}\ar[shift right]{r}[swap]{f_2} \&
      X_2 \ar[shift left]{r}{\beta_1}\ar[shift right]{r}[swap]{\beta_2} \&
      Y_2 \ar{r}{g_3}\ar{rrd}[swap]{\theta} \& Y_3\ar{r}{\gamma_3} \& Z_3 \\
      \& \& \& \& T\mathrlap{.}
    \end{tikzcd}
  \end{equation*}
  Si suponemos que \(\theta\colon Y_2\to T\) coiguala a lo que debe, entonces de
  la escisión común \(s\colon X_2\to X_1\) para \(f_1\) y \(f_2\) se deduce que
  \(\theta\beta_1=\theta\beta_2\). De manera similar, usando la escisión común
  \(t\colon Y_1\to X_1\) de \(\alpha_1\) y \(\alpha_2\) se deduce que
  \(\theta g_1=\theta g_2\). Por lo que estamos suponiendo se tiene que
  \(\theta\) se factoriza a través de \(\gamma_3 g_3\). La unicidad se sigue de
  que \(g_3\) y \(\gamma_3\) son epis.

  También se puede demostrar que la condición implica que el cuadrado de abajo a
  la derecha es un coproducto fibrado. Además, la condición se sigue de que
  \(g_3\) y \(\gamma_3\) son coigualadores.
\end{comentario}


\section{Teoría de gavillas}
\label{sec:0.2}

\section{Topologías de Grothendieck}
\label{sec:0.3}

\section{Teorema de Giraud}
\label{sec:0.4}

\section*{Ejercicios 0}
\label{sec:ejercicios-0}
\addcontentsline{toc}{section}{Ejercicios 0}