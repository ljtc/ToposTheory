\chapter{Preliminares}
\label{cap:0}

\section{Teoría de categorías}
\label{sec:0.1}

Esta sección no tiene la intención de dar una introducción detallada a las ideas
básicas de la teoría de categorías; nuestra intensión más bien es indicar
algunos de los conceptos y teoremas que serán asumidos como familiares, y
establecer algunas nociones estándar. El lector que no se considere familiarizado
con estos conceptos básicos será bien aconsejado a consultar el excelente libro
de Mac~Lane \pend{[CW]}, o cualquier otro texto estándar sobre teoría de
categorías antes de proceder más adelante con este libro.

Normalmente usaremos letras mayúsculas en script 
(\(\lcatC\), \(\lcatD\), \(\lcatE\),\dots) para denotar categorías
\enquote{grandes}. Cuando afirmemos \enquote{\(\lcatC\) es una categoría} sin
ningún calificativo adicional, se entenderá que \(\lcatC\) es un modelo de la
\enquote{teoría elemental de categorías} \pend{[72]}, es decir, \(\lcatC\) es
una \emph{metacategoría} en el sentido de \pend{[CW], capítulo I}. Esto
significa que no pensamos a \(\lcatC\) siendo formalmente definida dentro de un
modelo particular de la teoría de conjuntos; en particular, si \(X\) y \(Y\) son
objetos de \(\lcatC\) no pedimos que los morfismos de \(X\) a \(Y\) en
\(\lcatC\) formen un conjunto.

Sin embargo, a excepción del capítulo~\ref{cap:9}, normalmente asumiremos que
tenemos un modelo (fijo) de la teoría de conjuntos (incluyendo el axioma de
elección cuando sea necesario); y usaremos la letra \(\con\) para denotar la
\emph{categoría de conjuntos y funciones} que obtenemos de él. Usamos el término
\emph{categoría pequeña} para una categoría cuyos morfismos forman un conjunto.
Si \(\catC\) es una categoría pequeña, escribimos \(\pge{\catC}\) para la
categoría de \emph{pregavillas} sobre \(\catC\), es decir, funtores
contravariantes de \(\catC\) en \(\con\); entre los objetos de \(\pge{\catC}\),
tenemos los funtores \emph{representables} \(\gy_U\), donde \(U\) es un objeto
de \(\catC\), definido como \(\gy_U(V)=\hom_{\catC}(V,U)\). (Por razones
tipográficas, a veces escribiremos \(\gy(U)\) en lugar de \(\gy_U\); además
escribiremos \(\gy^U\) para el funtor covariante representable
\(\hom_{\catC}(U,-)\).) Frecuentemente haremos uso de los siguientes dos
resultados:

\begin{lemma}[label={0.11},note={Yoneda \pend{[187]}},short-note={Yoneda}]
  Dados \(U\) y \(X\) objetos de \(\catC\) y \(\pge{\catC}\) respectivamente,
  hay una biyección (natural en las dos variables) entre morfismos \(\gy_U\to X\)
  en \(\pge{\catC}\) y elementos del conjunto \(X(U)\).
\end{lemma}

\begin{lemma}
  Cualquier objeto de \(\pge{\catC}\) se puede expresar como un colímite de un
  diagrama cuyos vértices son funtores representables.
\end{lemma}
\begin{proof}
  Sea \(X\) un objeto de \(\pge{\catC}\), y sea \((\comma{\catC,X})\) la
  \emph{categoría coma} (pequeña) cuyos objetos son pares \(U,\alpha\), con
  \(U\) un objeto de \(\catC\) y \(\alpha\colon\gy_U \to X\) en \(\pge{\catC}\),
  y cuyos morfismos son triángulos conmutativos
  \begin{equation*}
    \begin{tikzcd}
      \gy_U \ar{rr}\ar{rd} && \gy_V\ar{ld} \\
      & X
    \end{tikzcd}
  \end{equation*}
  en \(\pge{\catC}\).

  Así, tenemos un diagrama obvio \((\comma{\catC,X})\to\pge{\catC}\) obvio que
  \enquote{olvida} dado por \((U,\alpha)\mapsto\gy_U\); se puede ver que el
  colímite de este diagrama es \(X\).
\end{proof}


\section{Teoría de gavillas}
\label{sec:0.2}

\section{Topologías de Grothendieck}
\label{sec:0.3}

\section{Teorema de Giraud}
\label{sec:0.4}

\section*{Ejercicios 0}
\label{sec:ejercicios-0}
\addcontentsline{toc}{section}{Ejercicios 0}