\chapter{Introducción}
\label{chap:intro}

La teoría de topos tiene sus orígenes en dos líneas separadas del desarrollo
matemático, las cuales permanecieron separadas durante diez años. Para tener una
apreciación balanceada de la importancia del tema, creo necesario
considerar la historia de estas dos líneas, y entender porque se unieron cuando
lo hicieron. Por lo tanto, inicio esta introducción con un pasaje histórico
(personal, y sin duda muy sesgado).

La primera de las dos líneas comienza con el surgimiento de la \emph{teoría de
gavillas}, originada en 1945 por J. Leray, desarrollada por H. Cartan y A. Weil
entre otros, y culminando en el trabajo publicado por J. P. Serre \pend{[107]}, 
A. Grothendieck \pend{[42]} y R. Godement \pend{[TF]}. Como una buena parte del
álgebra homológica, la teoría de gavillas originalmente fue concebida como una
herramienta para la topología algebraica, para axiomatizar la noción de
\enquote{sistema local de coeficientes} el cual era esencial para una buena
teoría de cohomología de espacios que no son simplemente conexos; y el título
completo del libro de Godement índica que aún era vista así en 1958. Sin
embargo, antes de esta fecha, la potencia de la teoría de gavillas había sido
reconocida por geómetras algebraicos y analíticos; y en años más recientes, su
influencia se ha extendido a muchas otras áreas de las matemáticas. (Para tener
dos ejemplos bastante diferentes ver \pend{[49]} y \pend{[106]}.)

Sin embargo, en geometría algebraica fue descubierto rápidamente que la noción
topológica de gavilla no era del todo adecuada, ya que la única topología
disponible en variedades algebraicas abstractas o esquemas, la topología Zariski,
no tenía \enquote{suficientes abiertos} para proveer de una buena noción
geométrica de localización. En su trabajo sobre técnicas de descenso
\pend{[43]} y el grupo fundamental étale \pend{[44]}, A. Grothendieck observó
que reemplazar \enquote{inclusión abierta de Zariski} por \enquote{morfismo
étale} era un paso en la dirección correcta; pero desafortunadamente los
esquemas que son étale sobre un esquema dado en general no forman un conjunto
parcialmente ordenado. Fue entonces necesario inventar la noción de
\enquote{topología de Grothendieck} sobre una categoría arbitraria, y la noción
generalizada de gavilla para tal topología para dar un marco para el desarrollo
de la cohomología étale.

Este marco fue construido durante el \enquote{Seminaire de Géométrie Algébrique
du Bois Marie} durante 1963--64 por Grothendieck con la asistencia de M. Artin,
J. Giraud, J. L. Verdier, y otros. (Las actas de este seminario fueron
publicadas en una versión muy alargada \pend{[GV]}, incluyendo algunos
resultados notables de P. Deligne, ocho años después.) Entre los resultados más
importantes del seminario original fue el teorema de Giraud, que muestra que las
categorías de gavillas generalizadas que surgen de esta manera pueden ser
completamente caracterizadas por propiedades de exactitud y condiciones de
tamaño; a la vista de este resultado, rápidamente se hizo evidente que estas
categorías de gavillas eran un tema de estudio más importante que los sitios (=
categorías + topologías) que les dan origen. A la vista de esto y dado que una
categoría con una topología puede ser vista como un \enquote{espacio topológico
generalizado}, el (ligeramente desafortunado) nombre de \emph{topos} fue dado a
las categorías que satisfacen los axiomas de Giraud.

No obstante, los topos seguían considerándose como vehículos primarios para
acarrear teorías de cohomología; no sólo cohomología étale, sino también la
\enquote{fppf}, cohomologías cristalinas, entre otras. La potencia de la
maquinaria desarrollada por Grothendieck fue ampliamente demostrada por los
sustanciales resultados geométricos obtenidos usando estas teorías de
cohomología en los años siguientes, culminando en la demostración de P. Deligne
\pend{[149]} de las famosas \enquote{conjeturas de Weil} ---el análogo \(\mod p\) de
la hipótesis de Riemann---. La maquinaria en sí fue desarrollada aún más, por
ejemplo en el trabajo de J. Giraud \pend{[38]} en cohomología no abeliana. Pero
el significado completo de la sentencia \enquote{el topos es más importante que
el sitio} parece que nunca fue apreciado por la escuela de Grothendieck. Por
ejemplo, aunque eran conscientes de la estructura cartesiana cerrada de los
topos (\pend{[GV, IV 10]}), nunca explotaron al máximo esta idea siguiendo las líneas
marcadas por Eilenberg y Kelly \pend{[160]}. Fue, por lo tanto, necesario que una
segunda línea de desarrollo proveyera el ímpetu para la teoría elemental de
topos.

El punto de partida de esta segunda línea se considera generalmente que es el
artículo pionero de F. W. Lawvere de 1964 sobre la teoría elemental de la
categoría de conjuntos \pend{[71]}. Sin embargo, considero que es necesario ir
un poco más atrás, a la demostración del teorema del encaje de Lubkin, Heron
Freyd y Mitchell para categorías abelianas \pend{[AC]}. Fue este teorema el
cual, mostrando que hay un conjunto explícito de axiomas elementales que
implican todas las propiedades de exactitud (finitas) de categorías de módulos,
pavimento el camino para un verdadero desarrollo autónomo de la teoría de
categorías como fundamento de las matemáticas.

(Casualmente el teorema del encaje de Freyd y Mitchell se considera
frecuentemente como una culminación en lugar de un punto de partida; esto es
porque me parece una mala interpretación (o al menos una inversión) de su
verdadero significado. Comunmente se piensa que dice \enquote{si quieres
demostrar algo acerca de una categoría abeliana, puedes asumir que es una
categoría de módulos}; mientras que yo creo que su verdadera importancia es
\enquote{si quieres demostrar algo acerca de categoría de módulos, puedes
trabajar en una categoría abeliana en general} ---el teorema del encaje asegura
que tu resultado será válido en esta generalidad, y olvidando la estructura
explícita de la categoría de módulos serás forzado a concentrarte en los
aspectos esenciales del problema---. Como ejemplo, compara la demostración
módulo teórica del lema de la serpiente en \pend{HA} con la demostración en
categorías abelianas en \pend{CW})