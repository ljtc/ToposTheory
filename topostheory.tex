\documentclass[11pt]{memoir}
\usepackage[spanish, mexico, es-noindentfirst]{babel}
\usepackage{mycats}
\usepackage[predefined={parent=section}]{keytheorems}
\usepackage[style=authoryear,giveninits=true]{biblatex}
\usepackage[style=mexican]{csquotes}
\usepackage[tracking=true,protrusion=true]{microtype}
\usepackage{comment}

\usepackage[colorlinks, allcolors=blue!50!violet]{hyperref}

%bibliografía
%\addbibresource{tt.bib}

%glosarios
\let\printindex\relax
\usepackage[symbols,index,abbreviations,automake]{glossaries-extra}
\newglossary*{nom}{Lista de Nombres}
\makeglossaries
\loadglsentries{gloss}

%geometría de la página
\settypeblocksize{196.85mm}{33pc}{*}
\setulmargins{4cm}{*}{*}
\setlrmargins{31.75mm}{*}{*}
\setmarginnotes{17pt}{86pt}{\onelineskip}
\setheadfoot{\onelineskip}{2\onelineskip}
\setheaderspaces{*}{2\onelineskip}{*}
\checkandfixthelayout

%fuentes
\setmainfont{Stix Two Text}
\setmathfont{Stix Two Math}
\setmathfont{Stix Two Math}[range=scr,StylisticSet=1]
\setmathfont{Erewhon Math}[range=\mscrS]

%punta de las flechas
\tikzcdset{arrow style=math font}
\tikzcdset{every arrow/.append style={line width=0.068em}}%ancho
\pgfmathdeclarefunction*{axis_height}{0}{%
    \begingroup%      
      \pgfmathreturn0.259em % línea de base
    \endgroup}% 

%comentarios que se activan con una etiqueta
\newif\ifshow%
\showtrue%cambiar a false para que desaparezcan los addendum
%comentario
\newcounter{com}
\ifshow%
\NewDocumentEnvironment{comentario}{+b}
  {% begin code
    \par\color{blue!60}\sffamily\noindent
    \refstepcounter{com}%
    \textbf{Comentario \thecom.}%
    \par\noindent\ignorespaces%
    #1
  }
  {% end code
    \par%
    \noindent\ignorespacesafterend%
  }
\else
\excludecomment{addendum}
\fi

%notación que se usa en el libro
%Aquí se pondrán los comandos para la notación del libro topos theory

%Capítulo 0
\newcommand*{\gy}{h}                        %funtor de Yoneda Grothendieck
\defset{\symfrak}{twocat#1}{A,B,C,D}        %2-categorías
\defset{\symbb}{mon#1}{A,B,C,D,E,F,G,H}     %mónadas
\NewDocumentCommand{\talg}{ m m }{#1^{#2}}  %álgebras
\NewDocumentCommand{\coalg}{ m m }{#1_{#2}} %coálgebras


%para recordar pendientes
\newcommand{\pend}[1]{{\color{red!70}#1}}

\setcounter{chapter}{-1}

\includeonly{capitulos/cap0}

\begin{document}

\frontmatter
\chapter*{Prefacio}
\label{chap:prefacio}
\addcontentsline{toc}{chapter}{Prefacio}

\tableofcontents*
\chapter*{Introducción}
\label{cap:intro}
\addcontentsline{toc}{chapter}{Introducción}

La teoría de topos tiene sus orígenes en dos líneas separadas del desarrollo
matemático, las cuales permanecieron separadas durante diez años. Para tener una
apreciación balanceada de la importancia del tema, creo necesario
considerar la historia de estas dos líneas, y entender porque se unieron cuando
lo hicieron. Por lo tanto, inicio esta introducción con un pasaje histórico
(personal, y sin duda muy sesgado).

La primera de las dos líneas comienza con el surgimiento de la \emph{teoría de
gavillas}, originada en 1945 por J. Leray, desarrollada por H. Cartan y A. Weil
entre otros, y culminando en el trabajo publicado por J. P. Serre \pend{[107]}, 
A. Grothendieck \pend{[42]} y R. Godement \pend{[TF]}. Como una buena parte del
álgebra homológica, la teoría de gavillas originalmente fue concebida como una
herramienta para la topología algebraica, para axiomatizar la noción de
\enquote{sistema local de coeficientes} el cual era esencial para una buena
teoría de cohomología de espacios que no son simplemente conexos; y el título
completo del libro de Godement índica que aún era vista así en 1958. Sin
embargo, antes de esta fecha, la potencia de la teoría de gavillas había sido
reconocida por geómetras algebraicos y analíticos; y en años más recientes, su
influencia se ha extendido a muchas otras áreas de las matemáticas. (Para tener
dos ejemplos bastante diferentes ver \pend{[49]} y \pend{[106]}.)

Sin embargo, en geometría algebraica fue descubierto rápidamente que la noción
topológica de gavilla no era del todo adecuada, ya que la única topología
disponible en variedades algebraicas abstractas o esquemas, la topología Zariski,
no tenía \enquote{suficientes abiertos} para proveer de una buena noción
geométrica de localización. En su trabajo sobre técnicas de descenso
\pend{[43]} y el grupo fundamental étale \pend{[44]}, A. Grothendieck observó
que reemplazar \enquote{inclusión abierta de Zariski} por \enquote{morfismo
étale} era un paso en la dirección correcta; pero desafortunadamente los
esquemas que son étale sobre un esquema dado en general no forman un conjunto
parcialmente ordenado. Fue entonces necesario inventar la noción de
\enquote{topología de Grothendieck} sobre una categoría arbitraria, y la noción
generalizada de gavilla para tal topología para dar un marco para el desarrollo
de la cohomología étale.

Este marco fue construido durante el \enquote{Seminaire de Géométrie Algébrique
du Bois Marie} durante 1963--64 por Grothendieck con la asistencia de M. Artin,
J. Giraud, J. L. Verdier, y otros. (Las actas de este seminario fueron
publicadas en una versión muy alargada \pend{[GV]}, incluyendo algunos
resultados notables de P. Deligne, ocho años después.) Entre los resultados más
importantes del seminario original fue el teorema de Giraud, que muestra que las
categorías de gavillas generalizadas que surgen de esta manera pueden ser
completamente caracterizadas por propiedades de exactitud y condiciones de
tamaño; a la vista de este resultado, rápidamente se hizo evidente que estas
categorías de gavillas eran un tema de estudio más importante que los sitios (=
categorías + topologías) que les dan origen. A la vista de esto y dado que una
categoría con una topología puede ser vista como un \enquote{espacio topológico
generalizado}, el (ligeramente desafortunado) nombre de \emph{topos} fue dado a
las categorías que satisfacen los axiomas de Giraud.

No obstante, los topos seguían considerándose como vehículos primarios para
acarrear teorías de cohomología; no sólo cohomología étale, sino también la
\enquote{fppf}, cohomologías cristalinas, entre otras. La potencia de la
maquinaria desarrollada por Grothendieck fue ampliamente demostrada por los
sustanciales resultados geométricos obtenidos usando estas teorías de
cohomología en los años siguientes, culminando en la demostración de P. Deligne
\pend{[149]} de las famosas \enquote{conjeturas de Weil} ---el análogo \(\mod p\) de
la hipótesis de Riemann---. La maquinaria en sí fue desarrollada aún más, por
ejemplo en el trabajo de J. Giraud \pend{[38]} en cohomología no abeliana. Pero
el significado completo de la sentencia \enquote{el topos es más importante que
el sitio} parece que nunca fue apreciado por la escuela de Grothendieck. Por
ejemplo, aunque eran conscientes de la estructura cartesiana cerrada de los
topos (\pend{[GV, IV 10]}), nunca explotaron al máximo esta idea siguiendo las líneas
marcadas por Eilenberg y Kelly \pend{[160]}. Fue, por lo tanto, necesario que una
segunda línea de desarrollo proveyera el ímpetu para la teoría elemental de
topos.

El punto de partida de esta segunda línea se considera generalmente que es el
artículo pionero de F. W. Lawvere de 1964 sobre la teoría elemental de la
categoría de conjuntos \pend{[71]}. Sin embargo, considero que es necesario ir
un poco más atrás, a la demostración del teorema del encaje de Lubkin, Heron
Freyd y Mitchell para categorías abelianas \pend{[AC]}. Fue este teorema el
cual, mostrando que hay un conjunto explícito de axiomas elementales que
implican todas las propiedades de exactitud (finitas) de categorías de módulos,
pavimento el camino para un verdadero desarrollo autónomo de la teoría de
categorías como fundamento de las matemáticas.

(Casualmente el teorema del encaje de Freyd y Mitchell se considera
frecuentemente como una culminación en lugar de un punto de partida; esto es
porque me parece una mala interpretación (o al menos una inversión) de su
verdadero significado. Comunmente se piensa que dice \enquote{si quieres
demostrar algo acerca de una categoría abeliana, puedes asumir que es una
categoría de módulos}; mientras que yo creo que su verdadera importancia es
\enquote{si quieres demostrar algo acerca de categoría de módulos, puedes
trabajar en una categoría abeliana en general} ---el teorema del encaje asegura
que tu resultado será válido en esta generalidad, y olvidando la estructura
explícita de la categoría de módulos serás forzado a concentrarte en los
aspectos esenciales del problema---. Como ejemplo, compara la demostración
módulo teórica del lema de la serpiente en \pend{HA} con la demostración en
categorías abelianas en \pend{[CW]}.)

Este teorema pronto fue seguido por el artículo de Lawvere \pend{[71]},
estableciendo una lista de axiomas elementales los cuales, agregando los axiomas
no elementales de completud y pequeñez local, son suficientes para caracterizar
a la categoría de conjuntos. (En un artículo subsecuente \pend{[72]}, Lawvere
provee una axiomatización similar para la categoría de categorías pequeñas, y D.
Schlomiuk \pend{[105]} hizo lo mismo para la categoría de espacios topológicos.)

Uno bien puede preguntar por qué este artículo no fue seguido inmediatamente por
la explosión de actividad que supuso la introducción de los topos elementales
seis años después. En retrospectiva, la respuesta es que los axiomas de Lawvere
son demasiado especializados: la categoría de conjuntos es un objeto
extremadamente útil como fundamento de las matemáticas, pero como objeto de
estudio axiomático no es (¡\textit{ritmo} de la actividad de Martin, Solovay, et.
al!) tremendamente interesante ---es demasiado \enquote{rígida} como para tener
una estructura interna---. De manera similar, si los axiomas de categorías
abelianas hubieran aplicado sólo a la categoría de grupos abelianos, y no a la
categoría de módulos o a la de gavillas abelianas, ellos también habrían sido
rechazados. Así, lo que se necesitaba para la categoría de conjuntos era una
axiomatización que también cubriera a las categorías de functores con valores en
conjuntos y categorías de gavillas con valores en conjuntos ---es decir, los
axiomas de topos elementales---.

En sus artículos subsecuentes (\pend{[73] y [75]}), Lawvere comienza a
investigar la idea de que el conjunto con dos elementos \(\set{\true, \false}\)
puede ser visto como un \enquote{objeto de valores de verdad} en la categoría de
conjuntos; en particular, observó que la presencia de tal objeto en una
categoría arbitraria nos permite reducir al axioma de comprensión a un enunciado
elemental acerca de funtores adjuntos. La misma idea es el corazón del trabajo
de H. Volger (\pend{[125] y [126]}) sobre lógica y categorías semánticas.

Mientras tanto, el lado del teorema del encaje de esto fue desarrollado por M.
Barr \pend{[2]}, quien formuló la  noción de \emph{categoría exacta} y la usó
como base para un teorema de encaje no aditivo. La noción cercanamente
relacionada de \emph{categoría regular} fue formulada independientemente por P.
A. Grillet \pend{[41]} y D. H. Van Osdol \pend{[122]}, quienes la usaron en sus
investigaciones sobre teoría general de gavillas; el mismo Barr obsrevó que el
teorema de Giraud puede ser visto como algo más que un caso especial de este
teorema de encaje. Esto quizás represente (lógicamente, si no cronológicamente)
el primer acercamiento de las dos líneas de desarrollo mencionadas antes.

Sin embargo, cerca del mismo tiempo la atención de Lawvere dió un giro hacia los
topos de Grothendieck; él observó que todo topos de Grothendieck tiene un objeto
de valores de verdad \(\Omega\), y que la noción de topología de Grothendieck
está cercanamente conectada con endomorfismos de \(\Omega\) (ver \pend{[LH]}).
Durante los años 1969--70, Lawvere y M. Tierney (quien antes había contribuido a
la teoría de categorías exactas) comenzaron a investigar las consecuencias de
tomar \enquote{existe un objeto de valores de verdad} como un axioma; el
resultado fue la teoría elemental de topos. Una proporsión notablemente grande
de la teoría básica fue desarrollada en ese período de 12 meses, como se hará
evidente del gran número de teoremas en los capítulos 1--4 de este libro en cuya
demostración se hará referencia a Lawvere y Tierney. 

Una vez que estos teoremas fueron conocidos por los matemáticos en general (es
decir, después de las lecturas de Lawvere en Zürich y Niece \pend{[LN]} en el
verano de 1970 y la conferencia en Dalhousie \pend{[LH]} en enero de 1971),
fueron inmediatamente aceptados y desarrollados por varias personas. Uno de los
primeros y más importantes fue P. Freyd, cuyas lecturas en University of New
South Wales \pend{[FK]} exploraron la teoría de encajes de topos; en
retrospectiva esto parece haber sido algo como un callejón sin salida, de que la
inversión del metateorema usual, mencionado arriba respecto a categorías
abelianas, aplica con incluso más fuerza en la teoría de topos ---ya que la gran
virtud de los axiomas de topos es su carácter elemental, uno no tiene que apelar
a un teorema de encaje no elemental para demostrar hechos elementales acerca de
topos. (El teorema de encaje de Freyd no será encontrado en este libro; pero la
parte más importante (y elemental) de él que muestra que todo topos puede ser
encajado en un topos booleano, es demostrado en~\ref{sec:7.5}). No obstante,
el trabajo de Freyd contiene muchos resultados técnicos importantes; en
particular su teorema de caracterización de objetos de números naturales es de
suma importancia.

Entre otros primeros trabajadores en teoría de topos, uno debe mencionar a J.
Bénabou y su estudiante J. Celeyrette en París \pend{[BC]}, y A. Kock y G. C.
Wraith en Arhaus \pend{[KW]}. C. J. Mikkelsen, un estudiante de Kock, fue el
primero en demostrar que uno de los axiomas de Lawvere y Tierney, el de
colímites finitos, se puede deducir de los otros; su tesis \pend{[84]} también
tiene muchas contribuciones importantes a la teoría de retículas en un topos.

En vista de la demostración de Lawvere y Tierney de la independencia de la
hipótesis del continuo \pend{[117]} se volvió un asunto importante determinar de
manera precisa la relación entre la teoría de topos elemental y la teoría de
conjuntos. La respuesta fue encontrada independientemente por J. C. Cole
\pend{[18]}, W. Mitchell \pend{[85]} y G. Osius \pend{[92]}. W. Mitchell también
introdujo una idea que se ha vuelto central en el tema: que en todo topos surge
un lenguaje interno que puede ser usado para hacer enunciados \enquote{casi
conjuntistas} acerca de los objetos y morfismos del topos. Aunque la idea
original se debe a Mitchell, su defensor más entusiasta ha sido indudablemente
J. Bénabou, sus estudiantes han usado extensivamente el lenguaje interno en años
recientes.

El siguiente avance importante fue hecho por R. Diaconescu, un estudiante de
Tierney cuya tesis fue completada en 1973. El teorema de Diaconescu \pend{[30]}
fue importante no sólo que le dió a la estructura de \(2\)-categoría a \(\cattop\)
sino porque también representó el primer uso importante de la teoría de
categorías interna. (Esta teoría se ha desarrollado a lo largo de los años de
manera azarosa, mayormente en trabajo no publicado de J. Bénabou.) Como bis,
Diaconescu demostró el teorema de Giraud relativo; el mismo Giraud \pend{[39]}
había demostrado una versión relativa de su teorema (por medios no elementales)
para topos de Grothendieck, y W. Mitchell formuló la forma elementalmente
correcta. Sin embargo, Mitchell sólo pudo demostrar esto en el caso especial
cuando el \enquote{objeto de generadores} (ver~\ref{4.43}) es \(1\); resultó que
el teorema de Diaconescu fue la herramienta esencial para demostrar el caso
general. Cerca del mismo tiempo, P. T. Johnstone \pend{[52]} también usó
categorías internas para demostrar que la construcción de Grothendieck del
funtor gavilla asociada se puede hacer en términos elementales.

El siguiente desarrollo (que de hecho se solapa con los anteriores) fue el
surgimiento de la noción de topos como teorías y el concepto de topos
clasificante. En cierto sentido, esto regresa al trabajo de Lawvere \pend{[176]}
sobre teorías algebraicas, pero su conexión con teoría de topos inició con el
trabajo de M. Hakim \pend{[45]}, una estudiante de Grothendieck, sobre esquemas
relativos  en el cual construyó el clasificador para anillos y anillos locales,
y estableció sus propiedades fundamentales. En 1972 A. Joyal y G. E. Reyes
\pend{[RM]} aislaron la noción de \enquote{teoría coherente} (= teoría
geométrica finitaria, en nuestra terminología), y demostraron que toda teoría de
ese tipo tiene un topos clasificante; su trabajo fue extendido después por Reyes
y M. Makkai \pend{[82]} cubriendo teorías geométricas infinitarias.

Fue F. W. Lawvere \pend{[LB]} quien observó primero que, dado el trabajo de
Reyes y Joyal, el teorema de Deligne sobre puntos de topos coherentes era
precisamente equivalente al teorema de completud de Gödel y Henkin para teorías
geométricas finitarias; y también Lawvere conjeturó el \enquote{teorema de
completud booleano valuado} para teorías infinitarias cuya demostración en
teoría de topos fue hecha por M. Barr \pend{[4]}.

Una vez más, el teorema de Diaconescu proveyó la clave para la
\enquote{relativización} de los resultados de Joyal y Reyes; el paso decisivo
fue dado en 1973 por G. C. Wraith, quien construyó un objeto clasificador sobre
un topos arbitrario con objeto de números naturales. De ahí el teorema general
de existencia de topos clasificantes era poco más que una formalidad; fue
logrado independientemente por A. Joyal, M. Tierney \pend{[119]} y J. Bénabou
\pend{[8]}.

Esto pone al día nuestro viaje histórico, al menos en lo que concierne a los
resultados más importantes. Ahora consideremos la posición presente de la teoría
de topos, y sus perspectivas futuras.

La primera cosa que debe ser dicha es que la parte básica de la topos
elementales (es decir, el contenido de los capítulos 1--5 de este libro) parece
estar completamente terminada. De hecho, sólo sé de una pregunta abierta
sustancial que surge de estos cinco capítulos (a saber, la existencia de pseudo
colímites finitos en \(\cattop\) vistos en la sección~\ref{sec:4.2}); sin duda
hay muchos otros puntos menores que deben ser clarificados, y muchos teoremas
cuyas demostraciones serán mejoradas y simplificadas con el tiempo, pero los
fundamentos del tema parecen estar firmemente establecidos. Esto es por supuesto
algo malo: es vital para la salud de un tema su continua revisión y
mejoramiento, y estoy incómodamente consiente de que escribiendo este libro he
contribuido mayormente  a la concretación de estos fundamentos. Mi una defensa
en contra de este cargo es que de cualquier forma la solidificación está
sucediendo, y es mejor que sea impresa que en folklore no publicado accesible
sólo a iniciados.

El matemático promedio, quien considera la teoría de categorías como
\enquote{abstracción sin sentido generalizada}, tiende a considerar la teoría de
topos teoría de categorías abstracta generalizada. (No hay duda de que ha
heredado esta reputación de su padre, la aproximación de Grothendieck a la
geometría algenraica.) Sin embargo S. Mac~Lane \pend{[179]} considera el
surgimiento de la teoría de topos como un \textit{decline} de abstracción en
teoría de categorías, y en álgebra abstracta en general. Estoy convencido de que
Mac~Lane eta en lo correcto, y de que su visión señala el camino al futuro más
probable del desarrollo de la teoría de topos; en casi todo el trabajo
\textit{reciente} de importancia en teoría de topos no conciernen a los topos
como un área abstracta y asilada de las matemáticas, sino a los topos como ayuda
al entendimiento y clarificación de los conceptos en otras áreas. (ver, por
ejemplo, \pend{[36], [57], [63], [79], [88], [90], [112], [130]}.)

Para tomar un ejemplo específico considera el teorema general de existencia de
topos clasificantes (\ref{6.56}). Una primera reacción al ver este teorema es
admirar su elegancia y generalidad; la segunda reacción (la cual viene mucho
tiempo después) es darse cuenta de su fundamental inutilidad ---una cualidad
que, por cierto, comparte con el teorema general del funtor adjunto---. Ya que
el único posible uso de tal teorema es reducir el estudio de una teoría
geométrica particular al estudio de su modelo genérico (o a la inversa, reducir
el estudio de un topos particular a la teoría del modelo genérico que tiene), y
el teorema como es demostrado en la sección~\ref{sec:6.5} simplemente no provee
un medio efectivo para pasar de una a la otra. Por tanto la demostración
\enquote{sintáctica} del mismo teorema en la sección~\ref{sec:7.4}, aunque es
apreciablemente más desordenada, es más valiosa en la práctica ---y es esta
demostración, no la otra en el capítulo anterior, la que mayoría del trabajo
siguiente en el tema.

Al decir que el futuro de la teoría de topos recae en la clarificaciónde otras
áreas de las matemáticas a través de la aplicación de las ideas de teoría de
topos, no quiero implicar que, como Grothendieck, veo a la teoría de topos como
una máquina de demolición de problemas abiertos en geometría algebraica o en
cualquier otro lado. Por el contrario, creo que es poco probable que la teoría
de topos elemental por sí misma resuelva algún problema mayor en matemáticas,
pero creo que la dispersión de la perspectiva de la teoría de topos en muchas
áreas de actividad matemática inevitablemente llevará a un entendimiento más
profundo de las características reales de los problemas, el cual es un preludio
esencial para su correcta solución.

¿Cuál es, entonces, la perspectiva de la teoría de topos? Brevemente, consiste
en el rechazo de la idea de que hay un universo fijo de conjuntos
\enquote{constantes} dentro del cual las matemáticas pueden y deben ser
desarrolladas, y el reconocimiento de que la noción de \enquote{estructura
variable} puede ser manejada más convenientemente en un universo de conjuntos
\emph{continuamente variables} que por el método, tradicional desde el
surgimiento de la teoría de conjuntos abstracta, de considerar separadamente un
dominio de variación (es decir, un espacio topológico) y una sucesión de
estructuras constantes adjuntadas a los puntos de su dominio. En palabras de F.
W. Lawvere \pend{[LB]}, \enquote{Toda noción de constancia es relativa, siendo
derivada perceptual o conceptualmente como un caso límite de variación y el
indiscutido valor de tales nociones para aclarar variación siempre está limitado
por su origen. Esto aplica en particular a la noción de conjunto constante, y
explica por qué gran parte de la teoría intuitiva de conjuntos lleva de alguna
forma a la teoría de conjuntos variables}. Es esta generalización de la ideas
desde conjuntos constantes a variables la que cae en el corazón de la teoría de
topos; y el lector que lo mantiene en mente, como objetivo último, mientras lee
este libro, ganará una gran comprensión.

Ahora, algunas palabras acerca de aquellas cosas que no hice en este libro.

(1) En la definición de topos, he tomado cartesiana cerrada y la existencia de
\(\Omega\) como dos axiomas separados, en lugar de combinarlos un un sólo axioma
de objetos potencia como sugiere A. Kock \pend{[66]}. (La equivalencia del
axioma de Kock esta, sin embargo, cubierta en los ejercicios del
capítulo~\ref{cap:1}.) A un nivel práctico defendería esta decisión por dos
lados: (a) que hay un gran número de resultados en el libro (significativamente
en el capítulo~\ref{cap:2}) que usan sólo la estructura cartesiana cerrada y no
los axiomas completos de topos, y algunos (por ejemplo, el teorema~\ref{1.47})
donde las exponenciales y \(\Omega\) son usados de formas esencialmente
distintas en la misma demostración; y (b) que si uno toma la definición con
objetos potencia, se ve obligado (como en\pend{[WB]}) a continuar inmediatamente
con la demostración un tanto técnica de que esta definición implica ser
cartesiana cerrada, y se pone en peligro de perder a los lectores en este punto
crítico. En un nivel más filosófico, añadiría (c) que la definición mediante
objetos potencia es realmente teórico conjuntista en lugar de una definición
teórico categórica de topos, en el sentido que subordina la noción de
\enquote{función} a la de \enquote{subconjunto} mediante el aparato teórico
conjuntista de identificar funciones con sus gráficas. Una de las principales
propiedades de la teoría de categorías es que toma \enquote{morfismo} como
noción primitiva al mismo nivel (\emph{no}, por cierto, superior a) que
\enquote{objeto}; por lo tanto es correcto que la definición de topos deba
incluir la estructura cartesiana cerrada.

(2) No he introducido el lenguaje de Mitchell y Bénabou hasta que el libro está
algo avanzado, al final del capítulo~\ref{cap:5}. Sé que hay algunas personas
cuyo libro de texto ideal sobre teoría de topos empezaría con la definición y
sólo suficientes desarrollos de propiedades de exactitud para introducir el
lenguaje la correctud de su interpretación; a partir de entonces todas las
demostraciones se conducirían con el lenguaje formal. No estoy de acuerdo con
este enfoque; creo que es imposible apreciar la potencia del lenguaje de
Mitchell y Bénabou hasta que tengas algo de experiencia demostrando cosas sin él
(de hecho, este es técnicamente el único lugar en el libro donde he seguido
conscientemente un orden particular del material por razones pedagógicas en
lugar de lógicas). También está el punto de que la aproximación con el lenguaje
formal se rompe cuando se confronta con el teorema de Giraud relativo
(\ref{4.46}); mientras que el lenguaje de Mitchell y Bénabou es una herramienta
bastante poderosa en demostraciones en un sólo topos, no está bien adaptada a
demostraciones en las cuales tenemos que pasar hacia adelante y hacia atrás
entre dos topos mediante un morfismo geométrico. (Es posible que la demostración
de~\ref{4.46} pueda hacerse más corta usando el lenguaje de categorías localmente
internas, pero eso es un asunto diferente.)

(3) Ya he mencionado que el teorema de encaje de Freyd \pend{[FK]} no será
encontrado en este libro. En consecuencia, el concepto de Freyd de topos bien
punteado juega un rol menor; no es introducido hasta la sección~\ref{9.3}.

(4) No he incluido ninguna referencia al trabajo más reciente de Freyd (no
publicado aún) sobre la teoría de \emph{alegorías}. Esta teoría hace para la
categoría de conjuntos y relaciones lo que los topos hacen para conjuntos y
funciones; es sabido que Freyd  mantiene que esto provee una base más simple y
natural que la teoría de topos para muchas de las ideas desarrolladas en este
libro, pero personalmente no estoy convencido de esto.

(5) No he mencionado el trabajo hecho por D. Bourn \pend{[13]}, R. Street
\pend{[113], [114]} y otros, sobre el desarrollo de un análogo \(2\)-categórico de
la teoría de topos. Me parece que los fundamentos de esta teoría no han
alcanzado un estado suficientemente definitivo para su tratamiento en forma de
libro.

(6) Una generalización de la teoría de topos cuya omisión me hace arrepentirme
un poco es la noción de J. Penon de \emph{quasitopos} \pend{[99]}. Sin embargo,
siento que introducirla al principio este libro simplemente habría introducido
complicaciones extra en las demostraciones sin beneficios en la forma de
ejemplos adicionales bien conocidos; e introducirla más adelante habría
involucrado una gran cantidad de duplicación. Espero, sin embargo, que las notas
O. Wyler sobre quasitopos (prometidas en \pend{[130]}) ayuden a llenar esta
grieta.

(7) La frase \enquote{universo de Grothendieck} no aparece en ningún lugar de
este libro. Esto es intencional; he sido deliberadamente tan vago como sea
posible (excepto en la sección~\ref{sec:9.3}) acerca de las propiedades de la
teoría de conjuntos que estoy usando, ya que realmente no importa. La teoría de
topos es una teoría elemental, y sus teoremas principales no son ---o no deben
ser--- dependientes de axiomas recónditos de la teoría de conjuntos. (De hecho
soy miembro con todas las cuotas pagadas del Movimiento de Liberación Matemática
fundado por J. H. Conway \pend{[157]}.) Sin embargo, si soy presionado usaría un
tipo de teoría de conjuntos de Bernays y Gödel para tener una distinción entre
categorías pequeñas (conjuntos) y categorías grandes (clases propias); pero
también quiero considerar ciertas \(2\)-categorías \enquote{muy grandes}
(sustancialmente \(\catcat\) y \(\cattop\)) cuyos objetos son categorías
grandes. Si quisiera ser estrictamente formal acerca de esto, necesitaría
introducir al menos un universo de Grothendieck; pero como todos los enunciados
que quiero hacer acerca de \(\catcat\) y \(\cattop\) son (equivalentes a)
enunciados elementales, no hay necesidad \textit{real} de hacerlo. Con el fin de
mantener cierta respetabilidad en el ámbito de la teoría de conjuntos, me he
limitado a considerar gavillas sólo sobre sitios pequeños; esto tiene la
desventaja de que no podemos enunciar el teorema de Giraud en su forma más hábil
(una categoría es un topos de Grothendieck si y sólo si es equivalente a la
categoría de gavillas canónicas sobre sí mismo), pero por lo demás no es tan
molesto como los autores de \pend{[GV]} quieren hacernos creer.

Finalmente, tengo que exponer mi postura sobre la pregunta más controversial en
toda la teoría de topos: ¿como decir el plural de topos? El lector ya habrá
observado que uso el plural del inglés;\footnote{En esta traducción se ha usado
el plural que se ha convenido en español, topos.} lo hago porque (en sentido
matemático) la palabra topos no se deriva directamente de su raíz griega, sino
que es una formación en reversa de topología. No tengo nada más que decir al
respecto, excepto preguntar a aquellos topósofos\footnote{Estoy en deuda con
Miles Reid por sugerir los términos \enquote{topósofos} y \enquote{toposofía}; e
insisto a mis colegas topósofos que los adopten.} quienes prefieren decir topoi,
cuando ellos van a divagar en un día frío, ¿llevan provisiones de té caliente
con ellos en termoi?
\chapter*{Notas al Lector}
\label{chap:notas_al_lector}
\addcontentsline{toc}{chapter}{Notas al Lector}


\mainmatter
\chapter{Preliminares}
\label{cap:0}

\section{Teoría de categorías}
\label{sec:0.1}

Esta sección no tiene la intención de dar una introducción detallada a las ideas
básicas de la teoría de categorías; nuestra intensión más bien es indicar
algunos de los conceptos y teoremas que serán asumidos como familiares, y
establecer algunas nociones estándar. El lector que no se considere familiarizado
con estos conceptos básicos será bien aconsejado a consultar el excelente libro
de Mac~Lane \pend{[CW]}, o cualquier otro texto estándar sobre teoría de
categorías antes de proceder más adelante con este libro.

Normalmente usaremos letras mayúsculas en script 
(\(\lcatC\), \(\lcatD\), \(\lcatE\),\dots) para denotar categorías
\enquote{grandes}. Cuando afirmemos \enquote{\(\lcatC\) es una categoría} sin
ningún calificativo adicional, se entenderá que \(\lcatC\) es un modelo de la
\enquote{teoría elemental de categorías} \pend{[72]}, es decir, \(\lcatC\) es
una \emph{metacategoría} en el sentido de \pend{[CW], capítulo I}. Esto
significa que no pensamos a \(\lcatC\) siendo formalmente definida dentro de un
modelo particular de la teoría de conjuntos; en particular, si \(X\) y \(Y\) son
objetos de \(\lcatC\) no pedimos que los morfismos de \(X\) a \(Y\) en
\(\lcatC\) formen un conjunto.

Sin embargo, a excepción del capítulo~\ref{cap:9}, normalmente asumiremos que
tenemos un modelo (fijo) de la teoría de conjuntos (incluyendo el axioma de
elección cuando sea necesario); y usaremos la letra \(\con\) para denotar la
\emph{categoría de conjuntos y funciones} que obtenemos de él. Usamos el término
\emph{categoría pequeña} para una categoría cuyos morfismos forman un conjunto.
Si \(\catC\) es una categoría pequeña, escribimos \(\pge{\catC}\) para la
categoría de \emph{pregavillas} sobre \(\catC\), es decir, funtores
contravariantes de \(\catC\) en \(\con\); entre los objetos de \(\pge{\catC}\),
tenemos los funtores \emph{representables} \(\gy_U\), donde \(U\) es un objeto
de \(\catC\), definido como \(\gy_U(V)=\hom_{\catC}(V,U)\). (Por razones
tipográficas, a veces escribiremos \(\gy(U)\) en lugar de \(\gy_U\); además
escribiremos \(\gy^U\) para el funtor covariante representable
\(\hom_{\catC}(U,-)\).) Frecuentemente haremos uso de los siguientes dos
resultados:

\begin{lemma}[label={0.11},note={Yoneda \pend{[187]}},short-note={Yoneda}]
  Dados \(U\) y \(X\) objetos de \(\catC\) y \(\pge{\catC}\) respectivamente,
  hay una biyección (natural en las dos variables) entre morfismos \(\gy_U\to X\)
  en \(\pge{\catC}\) y elementos del conjunto \(X(U)\).
\end{lemma}

\begin{lemma}
  Cualquier objeto de \(\pge{\catC}\) se puede expresar como un colímite de un
  diagrama cuyos vértices son funtores representables.
\end{lemma}
\begin{proof}
  Sea \(X\) un objeto de \(\pge{\catC}\), y sea \((\comma{\catC,X})\) la
  \emph{categoría coma} (pequeña) cuyos objetos son pares \(U,\alpha\), con
  \(U\) un objeto de \(\catC\) y \(\alpha\colon\gy_U \to X\) en \(\pge{\catC}\),
  y cuyos morfismos son triángulos conmutativos
  \begin{equation*}
    \begin{tikzcd}
      \gy_U \ar{rr}\ar{rd} && \gy_V\ar{ld} \\
      & X
    \end{tikzcd}
  \end{equation*}
  en \(\pge{\catC}\).

  Así, tenemos un diagrama obvio \((\comma{\catC,X})\to\pge{\catC}\) obvio que
  \enquote{olvida} dado por \((U,\alpha)\mapsto\gy_U\); se puede ver que el
  colímite de este diagrama es \(X\).
\end{proof}


\section{Teoría de gavillas}
\label{sec:0.2}

\section{Topologías de Grothendieck}
\label{sec:0.3}

\section{Teorema de Giraud}
\label{sec:0.4}

\section*{Ejercicios 0}
\label{sec:ejercicios-0}
\addcontentsline{toc}{section}{Ejercicios 0}
\chapter{Topos Elementales}
\label{cap:1}

\section{Definición y ejemplos}
\label{sec:definicion_ejemplos}

\begin{definition}
  Test de definición.
\end{definition}

\section{Relaciones de equivalencia y morfismos parciales}
\label{sec:relaciones_equivalencia_morfismos_parciales}

\section{La categoría \(\toposE\op\)}
\label{sec:categoria_Eop}

\section{Funtores producto fibrado}
\label{sec:funtores_producto_fibrado}

\section{Factorización con imagen}
\label{sec:factorizacion_imagen}

\section*{Ejercicios 1}
\label{sec:ejercicios-1}
\addcontentsline{toc}{section}{Ejercicios 1}
\chapter{Teoría de Categorías Interna}
\label{chap:2}

\section{Categorías internas y diagramas}
\label{sec:2.1}

\section{Límites y colímites internos}
\label{sec:2.2}

\section{Diagramas en un topos}
\label{sec:2.3}

\section{Profuntores internos}
\label{sec:2.4}

\section{Categorías filtrantes}
\label{sec:2.5}

\section*{Ejercicios 2}
\label{sec:ejercicios-2}
\addcontentsline{toc}{section}{Ejercicios 2}

\chapter{Topología y gavillas}
\label{chap:3}

\section{Topología}
\label{sec:3.1}

\section{Gavillas}
\label{sec:3.2}

\section{El funtor gavilla asociada}
\label{sec:3.3}

\section{\(\gav{\toposE,j}\) como categoría de fracciones}
\label{sec:3.4}

\section{Ejemplos de topologías}
\label{sec:3.5}

\section*{Ejercicios 3}
\label{sec:ejercicios-3}
\addcontentsline{toc}{section}{Ejercicios 3}

\chapter{Morfismos geométricos}
\label{cap:4}

\section{El teorema de factorización}
\label{sec:4.1}

\section{La construcción de pegado}
\label{sec:4.2}

\section{Teorema de Diaconescu}
\label{sec:4.3}

\section{Morfismos acotados}
\label{sec:4.4}

\section*{Ejercicios 4}
\label{sec:ejercicios-4}
\addcontentsline{toc}{section}{Ejercicios 4}
\chapter{Aspectos lógicos de la teoría de topos}
\label{cap:5}

\section{Topos booleanos}
\label{sec:5.1}

\section{El axioma de elección}
\label{sec:5.2}

\section{El axioma (SG)}
\label{sec:5.3}

\section{El lenguaje de Mitchell y Bénabou}
\label{sec:5.4}

\section*{Ejercicios 5}
\label{sec:ejercicios-5}
\addcontentsline{toc}{section}{Ejercicios 5}
\chapter{Objetos de números naturales}
\label{cap:6}

\section{Definición y propiedades básicas}
\label{sec:6.1}

\section{Cardinales finitos}
\label{sec:6.2}

\section{El objeto clasificador}
\label{sec:6.3}

\section{Teorías algebraicas}
\label{sec:6.4}

\section{Teorías geométricas}
\label{sec:6.5}

\section{Objetos de números reales}
\label{sec:6.6}

\section*{Ejercicios 6}
\label{sec:ejercicios-6}
\addcontentsline{toc}{section}{Ejercicios 6}
\chapter{Teoremas de Deligne y Barr}
\label{chap:7}

\section{Puntos}
\label{sec:7.1}

\section{Topos espaciales}
\label{sec:7.2}

\section{Topos coherentes}
\label{sec:7.3}

\section{Teorema de Deligne}
\label{sec:7.4}

\section{Teorema de Barr}
\label{sec:7.5}

\section*{Ejercicios 7}
\label{sec:ejercicios-7}
\addcontentsline{toc}{section}{Ejercicios 7}
\chapter{Cohomología}
\label{chap:8}

\section{Definiciones básicas}
\label{sec:8.1}

\section{Cohomología de \v{C}ech}
\label{sec:8.2}

\section{Torsores}
\label{sec:8.3}

\section{Grupos fundamentales profinitos}
\label{sec:8.4}

\section*{Ejercicios 8}
\label{sec:ejercicios-8}
\addcontentsline{toc}{section}{Ejercicios 8}
\chapter{Teoría de topos y teoría de conjuntos}
\label{cap:9}

\section{Finitud de Kuratowski}
\label{sec:9.1}

\section{Objetos transitivos}
\label{sec:9.2}

\section{Teoremas de equiconsistencia}
\label{sec:9.3}

\section{La construcción filtro-potencia}
\label{sec:9.4}

\section{Independencia de la hipótesis del continuo}
\label{sec:9.5}

\section*{Ejercicios 9}
\label{sec:ejercicios-9}
\addcontentsline{toc}{section}{Ejercicios 9}

\appendix
\chapter{Categorías localmente internas}
\label{app}

\backmatter
%Impresión de glosarios
\printindex[title=Índice de términos]
\printsymbols[title=Lista de símbolos]
\printabbreviations[title=Acrónimos]
\printglossary[type=nom]

%impresión de bibliografías

\end{document}