%Aquí se pondrán los comandos para la notación del libro topos theory

%Introducción
\newcommand*{\cattop}{\symfrak{Top}}            %categoría de topos
\newcommand*{\catcat}{\symfrak{Cat}}            %categoría de categorías
\newcommand*{\betop}{\symfrak{BTop}/\toposE}    %topos acotados sobre E

%Capítulo 0
\newcommand*{\gy}{h}                            %funtor de Yoneda Grothendieck
\defset{\symfrak}{twocat#1}{A,B,C,D}            %2-categorías
\defset{\symbb}{mon#1}{A,B,C,D,E,F,G,H,K}       %mónadas
\NewDocumentCommand{\talg}{ m m }{#1^{#2}}      %álgebras
\NewDocumentCommand{\coalg}{ m m }{#1_{#2}}     %coálgebras
\renewcommand*{\con}{\toposS}                   %categoría de conjuntos
\renewcommand*{\id}{1}                          %morfismo identidad
\DeclareMathOperator{\coig}{coig}               %coigualador